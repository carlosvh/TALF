\documentclass[11pt]{article}
    \title{\textbf{Práctica 1}}
    \author{Carlos Velasco Hurtado}
    \date{}
    
    \addtolength{\topmargin}{-3cm}
    \addtolength{\textheight}{3cm}
    
    \usepackage{amsmath,amsthm}
    \usepackage{enumitem}
    \usepackage{nccmath}
    
\title{Teoría de Autómatas y Lenguajes Formales\\[.4\baselineskip]Práctica 2}
\author{Velasco Hurtado, Carlos}
\date{\today}
    
\begin{document}

\maketitle
\thispagestyle{empty}

\section*{Exercise 1}
Find the power set $R^3$  of $R=\{(1,1),(1,2),(2,3),(3,4)\}$. Check your answer with the script powerrelation.m and write a \LaTeX\, document with the solution step by step.
\\\\
\begin{equation*}
R = 
\begin{pmatrix}
1 & 1 & 0 & 0\\
0 & 0 & 1 & 0\\
0 & 0 & 0 & 1\\
0 & 0 & 0 & 0 
\end{pmatrix}
\end{equation*}

\begin{equation}
R^2 = R \times R = 
\begin{pmatrix}
1 & 1 & 0 & 0\\
0 & 0 & 1 & 0\\
0 & 0 & 0 & 1\\
0 & 0 & 0 & 0 
\end{pmatrix}
\times
\begin{pmatrix}
1 & 1 & 0 & 0\\
0 & 0 & 1 & 0\\
0 & 0 & 0 & 1\\
0 & 0 & 0 & 0 
\end{pmatrix}
=
\begin{pmatrix}
1 & 1 & 1 & 0\\
0 & 0 & 0 & 1\\
0 & 0 & 0 & 0\\
0 & 0 & 0 & 0 
\end{pmatrix}
\end{equation}
\begin{equation}
R^3 = R^2 \times R = 
\begin{pmatrix}
1 & 1 & 1 & 0\\
0 & 0 & 0 & 1\\
0 & 0 & 0 & 0\\
0 & 0 & 0 & 0 
\end{pmatrix}
\times
\begin{pmatrix}
1 & 1 & 0 & 0\\
0 & 0 & 1 & 0\\
0 & 0 & 0 & 1\\
0 & 0 & 0 & 0 
\end{pmatrix}
=
\begin{pmatrix}
1 & 1 & 1 & 1\\
0 & 0 & 0 & 0\\
0 & 0 & 0 & 0\\
0 & 0 & 0 & 0 
\end{pmatrix}
\end{equation}
Using the script powerrelation.m, the solution can be found with the command:\\
$>>$ powerrelation({['1','1'],['1','2'],['2','3'],['3','4']},3)\\
which returns [1,1] = 11 [1,2] = 12 [1,3] = 13 [1,4] = 14.

\end{document}

