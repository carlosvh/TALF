\documentclass[11pt]{article}
	\title{Teoría de Autómatas y Lenguajes Formales\\[.4\baselineskip]Práctica 3: Máquina de Turing, Funciones Recursivas y Lenguaje WHILE}
    \author{Velasco Hurtado, Carlos}
    \date{\today}
    
    \addtolength{\topmargin}{-3cm}
    \addtolength{\textheight}{3cm}
	\usepackage{whilecode2}
	\usepackage{graphicx}
	\usepackage{amsmath}

\begin{document}

\maketitle
\thispagestyle{empty}

\section{Exercise 1}
Define the TM solution of exercise 3.4 of the problem list and test its correct behaviour.
\\

\includegraphics[height=8cm]{MT-suma.jpg}

After testing its behaviour with JFLAP, we can confirm that it works correctly. However, there's one thing to notice. In the transition from q1 to q2 the instruction reads a * symbol when it should read the blank symbol. It proved imposible to me to test the behaviour of the TM using a blank symbol as a separator, as I couldn't write it in any way, so I replaced it with a * symbol.
\\\\
Another thing to note is that this TM starts and finishes to the left of the first input string.

\newpage
Matrix definition of this TM:
\[
\begin{bmatrix}
    q0 & * & r & q0 \\
    q0 & | & r & q1 \\
    q1 & * & | & q2 \\
    q1 & | & r & q1 \\
    q2 & * & l & q3 \\
    q2 & | & r & q2 \\
    q3 & * & l & q4 \\
    q3 & | & * & q3 \\
    q4 & * & h & q4 \\
    q4 & | & l & q4 \\
\end{bmatrix}
\]\\

\newpage

\section{Exercise 2}

Define a recursive function for the sum of three values.
\\

$<<\pi _1^1|\sigma(\pi_3^3)>|\sigma(\pi_4^4)>$ \\
%<<\π^1_1|σ(π^3_3)>|σ(π^4_4)>
\includegraphics[scale = 0.5]{evalRec.png}


\section{Exercise 3}
Implement a WHILE program that computes the sum of three values.

\begin{whilecode}[H]

 \While{$X1 \not = 0$}{

  $X1 \Assig X1 - 1$\;
  $X4 \Assig X4 + 1$
 }
 \While{$X2 \not = 0$}{
  $X2 \Assig X2 - 1$\;
  $X4 \Assig X4 + 1$
 }
 \While{$X3 \not = 0$}{
  $X3 \Assig X3 - 1$\;
  $X4 \Assig X4 + 1$
 }
 X1 \Assig X4;

\end{whilecode}

\end{document}

